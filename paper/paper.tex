\documentclass[9pt,letterpaper]{extarticle}
\usepackage{extsizes}
\usepackage[paper=letterpaper,margin=1in]{geometry}

% Table coloring: Must be imported before TikZ
\usepackage[table,x11names,dvipsnames]{xcolor}

\usepackage{xspace,amsmath,amsfonts,amssymb,hyperref,tikz,multirow}
\usepackage{graphicx}
% \graphicspath{{../data/R_plots/}}

\usepackage{parskip}

\newcommand{\todo}[1]{%
	\mbox{}% prevent marginpar from being on previous paragraph
	\marginpar{%
		\colorbox{red!80!black}{\textcolor{white}{to-do}}%
		\vspace*{-22pt}% hack!
	}%
	\textcolor{red}{#1}%
}

\usepackage{algorithmicx}
\usepackage{algorithm}
\usepackage[noend]{algpseudocode}

\usepackage{graphicx}
\usepackage{multicol}
\usepackage{epigraph}

\title{Final Tagless While Language, Something Quippy Here}
\author{Jeff Young}
\date{}

  % \epigraph{And we have ascended, to a new plane of enlightenment. We no longer
  %   need material things to incant, nor are we bound by the physical realm. We
  %   have, in a sense, become ephemeral, tagless, in all respects. A gossamer, as
  %   flimsy as our attachments to this realm.}
\begin{document}
	\maketitle
	% \begin{abstract}
  %   Do we want/need an abstract?
	% \end{abstract}
  \section{System Overview}
  % What was the Technique that was used
  % What is the Core System and What are the Extensions to the Core system
  % Alteast two dimensions of extensions, a New Case, and New Operations, on all cases
  This project implement's the WHILE Language using the final tagless style in
  just two haskell files, ``CoreLang.hs'' and ``Lang.hs''. Final tagless style,
  while briefly discussed in class, is essentially a semantic-based
  implementation of whatever is being made. In Haskell, this translates into a
  typeclass based approach for the WHILE language. For example, instead of
  having a Abstract data type for Statements in the language, we will instead
  have a typeclass with functions that represent statements in our language. In
  this sense, Final Tagless, allows extensibility in the constructor dimension,
  because one may extend value constructors and make a new instantiation of the
  type class, and final tagless allows extensibility in the operator dimension,
  because one is free to write their own typeclass in final tagless style and
  instantiate. An important note is that final tagless style is also extensible
  in a third dimensions, the evaluator dimension, because all of the language
  constructs are functions in a typeclass, one is free to implement them how
  they please. Hence, one may choose to implement them with strict evaluation,
  or lazy evaluation, with a global state or not.

  \subsection{The Core System}
  The core system typeclass definitions are to be found in the ``CoreLang.hs''
  file. The file holds all typeclass definitions for the core language, and all
  extensions, it is separated into four distinct parts. The first part, labeled
  ``Core Language'' in comments, consists of the typeclasses that are required
  to support a minimal implementation of the WHILE language. Specifically, these
  are boolean expressions, arithmetic expressions, and statements; all of their
  types are self-explanatory. The while language, in it's core form, only
  supports integers, booleans, and a unit as primitives. The definition of these
  primitives are located in the ``CoreLang'' labeled section of the ``Lang.hs''
  file, along with all class instantiations.

  \subsection{The Extensions to the Core}
  The extensions made to the core language are as follows. In the constructor
  domain I extend the language with Strings, Floats, and a list constructor. In
  the operator dimension I extend the language with two compatible extensions, a
  trivial extension, adding exponentials to the language, and adding a printing
  operator to the language. The harder, non-compatible extensions, include
  operators for strings, floats, and lists. In this way the core language is
  extended in both the constructor dimension, and operator dimensions. It should
  be noted that for every implementation I instantiate the core language with a
  state monad thereby ``extending'' the while language with a global state. This
  suggests that the final tagless approach is flexible enough to incorporate any
  monad or monad transformer stack. While I did not explore the ``third
  dimension'' of extensibility this project proves that it is possible. 

	\section{Extensibility Scenario}
  The Extensibility Scenario for the WHILE language is identical to that of any
  embedded DSL. The scenario is generally this: I, the EDSL author want to write
  my EDSL in a static, strongly typed language like haskell. Then I want to be
  able to share that EDSL with anyone on the internet and have them use it. This
  requires that I've taken time to consider how my users may use my EDSL. But it
  is impossible to predict every use case, so by writing my EDSL in a final
  tagless style I can ensure that people who do not have access to the source
  code (because I'll restrict access) can still extend and use the EDSL however
  they deem fit. In this project's specific case the core WHILE language is the
  EDSL, and some other developer on the internet realized they need a list
  construct, and also would like to print statements in the language to a
  command prompt. Because I have written my implementation of the WHILE language
  in the final tagless style they are freely able to add these features to the
  core if they so desire. This scenario is demonstrated in the 

  \section{Challenges and Design Decisions}
\end{document}