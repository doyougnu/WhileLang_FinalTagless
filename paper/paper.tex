\documentclass[9pt,letterpaper]{extarticle}
\usepackage{extsizes}
\usepackage[paper=letterpaper,margin=1in]{geometry}

% Table coloring: Must be imported before TikZ
\usepackage[table,x11names,dvipsnames]{xcolor}

\usepackage{xspace,amsmath,amsfonts,amssymb,hyperref,tikz,multirow}
\usepackage{graphicx}
% \graphicspath{{../data/R_plots/}}

\usepackage{parskip}

\newcommand{\todo}[1]{%
	\mbox{}% prevent marginpar from being on previous paragraph
	\marginpar{%
		\colorbox{red!80!black}{\textcolor{white}{to-do}}%
		\vspace*{-22pt}% hack!
	}%
	\textcolor{red}{#1}%
}

\usepackage{algorithmicx}
\usepackage{algorithm}
\usepackage[noend]{algpseudocode}

\usepackage{graphicx}
\usepackage{multicol}
\usepackage{epigraph}

\title{Final Tagless While Language, An Exercise in Mentioning the Unmentionable}
\author{Jeff Young}
\date{}

  % \epigraph{And we have ascended, to a new plane of enlightenment. We no longer
  %   need material things to incant, nor are we bound by the physical realm. We
  %   have, in a sense, become ephemeral, tagless, in all respects. A gossamer, as
  %   flimsy as our attachments to this realm.}
\begin{document}
	\maketitle
	% \begin{abstract}
  %   Do we want/need an abstract?
	% \end{abstract}
  \section{System Overview}
  % What was the Technique that was used
  % What is the Core System and What are the Extensions to the Core system
  % Alteast two dimensions of extensions, a New Case, and New Operations, on all cases
  This project implement's the WHILE Language using the final tagless style in
  just two haskell files, \underline{CoreLang.hs} and \underline{Extensions.hs}.
  Final tagless style, while briefly discussed in class, is essentially a
  semantic-based implementation of whatever is being made. In Haskell, this
  translates into a typeclass based approach for the WHILE language. For
  example, instead of having a Abstract data type for statements in the
  language, we will instead have a typeclass with functions that represent
  statements in our language. In this sense, final tagless allows extensibility
  in the constructor dimension, because one may extend value constructors and
  make a new instantiation of the type class, and it also allows extensibility
  in the operator dimension, because one is free to write their own typeclass in
  final tagless style, instantiate and mix it with the core system. An important
  note is that final tagless style is also extensible in a third dimension, the
  evaluator dimension, because all of the language constructs are functions in a
  typeclass, one is free to implement them how they please. Hence, one may
  choose to implement them with strict evaluation, or lazy evaluation, with a
  global state or not.

  \subsection{The Core System}
  The core system typeclass definitions are to be found in the
  \underline{CoreLang.hs} file. The file holds all typeclass definitions and
  instantiations for the core language. These typeclasses support a minimal
  implementation of the WHILE language. Specifically, these are boolean
  expressions, arithmetic expressions, and statements; all of their types are
  self-explanatory. The while language, in it's core form, only supports
  integers, booleans, and a unit as primitives.

  \subsection{The Extensions to the Core}
  The extensions made to the core language are in the \underline{Extensions.hs}
  file and are as follows. In the constructor domain I extend the language with
  Strings, Floats, and a list constructor. In the operator dimension I extend
  the language with two compatible extensions, a trivial extension, adding
  exponentials to the language, and adding a printing operator to the language.
  The harder, non-compatible extensions, include operators for strings, floats,
  and lists. In this way the core language is extended in both the constructor
  dimension, and operator dimensions. It should be noted that for every
  implementation I instantiate the core language with a state monad thereby
  ``extending'' the while language with a global state (which is normally
  assumed). This suggests that the final tagless approach is flexible enough to
  incorporate any monad or monad transformer stack. While I did not explore the
  ``third dimension'' of extensibility this project proves that it is possible.

	\section{Extensibility Scenario}
  The Extensibility Scenario for the WHILE language is identical to that of any
  embedded DSL. The scenario is generally this: I, the EDSL author want to write
  my EDSL in a static, strongly typed language like haskell. Then I want to be
  able to share that EDSL with anyone on the internet and have them use it. This
  requires that I've taken time to consider how my users may use my EDSL. But it
  is impossible to predict every use case, so by writing my EDSL in a final
  tagless style I can ensure that people who do not have access to the source
  code (because I'll restrict access) can still extend and use the EDSL however
  they deem fit. In this project's specific case the core WHILE language is the
  EDSL, and some other developer on the internet realized they need a list
  construct, and also would like to print statements in the language to a
  command prompt. Because I have written my implementation of the WHILE language
  in the final tagless style they are freely able to add these features to the
  core if they so desire. This scenario is demonstrated in the ``Trivial
  Extensions, and Experiments'' sections in the \underline{Extensions.hs} file.

  \section{Challenges and Design Decisions}
  \subsection{Using Final Tagless with Monads, A balancing act}
  For both the core and extended implementations of the WHILE language I make
  use of a state monad, holding a variable map as the state, and then
  instantiate the typeclasses using the state monad. The reader may have noticed
  that I awkwardly create a new data type like this to instantiate against:
  \begin{verbatim}
    -- | an eval monad
    data EvalOne a = E (State (VarStore Prims) Prims)
  \end{verbatim}

  Instead, I could do something like this, which is much cleaner:
  \begin{verbatim}
    -- | a cleaner eval monad
    type EvalTwo a = State (VarStore Prims) Prims
  \end{verbatim}

  Not only being more idiomatic haskell style, it maintains the phantom type
  variable and is explicit that I am really just creating a type synonym.
  However, if one tries to use this with the final tagless style, they will
  discover that it will not type check.

  First, it will not type check because type synonyms \textit{cannot be}
  partially applied for compiler reasons. Which means that everytime one uses a
  type synonym in haskell they must satisfy the kind of the type synonym.
  Normally this is not an issue because one rarely uses a type synonym to
  instantiate a type class. But in our case that is exactly what we are trying
  to do. Furthermore, the final tagless style \textit{depends} on the
  constructor used in the instantiation to be of kind $\star$ $\rightarrow$
  $\star$. EvalTwo is of kind $\star$ $\rightarrow$ $\star$, but can only be
  used in a type class instantiation as kind $\star$, because it cannot be
  partially applied. Thus, if we give the type constructor an ``a'' or anthing
  else in the instance declaration then it will be kind $\star$ (satisfying that
  the type synonym isn't partially applied), but that creates a type error for
  the type class because we are using constructor classes (which means the
  constructor must be kind $\star$ $\rightarrow$ $\star$). Thus, we can never
  create a class constructor for a type synonym because of this incongruity.

  To get around this we are forced to create a new abstract data type to wrap
  the state monad. This is inconvienient because then we must constantly
  box/unbox the monad in order to use it, which is exactly what is observed in
  either the core or extended implementation.

  Another restriction that I encountered mixing monads with the final tagless
  style is summed up in the following code:

  \begin{verbatim}
    var v = E e 
      where e = do
              st <- get
              return $ st M.! v
  \end{verbatim}

  Obviously this statement can be rewritten in a clean and elegant way using
  bind: 
  \begin{verbatim}
    var v = get >>= E . return . (M.! v)
  \end{verbatim}

  This form is much nicer and performs exactly the same, the problem is that it
  will not type check because we have wrapped our monad in a abstract data type
  yet have not made that type a monad instance. Thus, we will not type check and
  will incur errors saying there is not instance of MonadState for get, and
  Monad for return. While these instances could be added it is non-trivial (see
  below) and, I felt it was important to highlight that this is added
  boilerplate caused by the final tagless approach.

  \subsection{Lazy evaluation or the lack thereof}
  Consider the let statement implementation for the core language (it's
  identical to the extended version):
  \begin{verbatim}
  let_ v (E x) = E e
    where e = do
            st <- get
            x' <- x
            put $ M.insert v x' st
            return NoOp
  \end{verbatim}

  Looking at this code snippet one might begin to wonder if instead of strictly
  evaluating ``x''' to ``x prime''', I could implement lazy evaluation by changing
  the call to M.insert from inserting ``x prime''' to just inserting ``x'''. The
  answer is no, because to do so would cause a type error in the Variable
  Store (whose type for values is ``Prims'''). Basically we would be saving
  something of type ``r a''' in the map instead of type ``Prims'''. One may
  respond ``We'll just change the type of the Variable Store!'''. The suggestion
  would look like this:
  \begin{verbatim}
   -- | Map to hold let bound variables
   type VarStore a = M.Map String a
   
   -- | an eval monad
   data Eval a = E (State (VarStore (r a)) Prims) 
  \end{verbatim}

  Obviously this will not work. The first problem is that we have a type
  constructor ``r'' that is unbound, and therefore will cause a type error. But
  the only way to make a lazy evaluator for our final tagless WHILE language
  would be to add a statement, unevaluated, into the variable map. The only way
  to achieve the desired behavior is a definition like this:
  \begin{verbatim}
   -- | Map to hold let bound variables
   type VarStore a = M.Map String a
   
   -- | an eval monad
   data TEval a = TE (State (VarStore (TEval Prims)) Prims) 
  \end{verbatim}

  But this forces one to make the data type an instance of MonadState. I'm not
  sure if it is possible to write this instance as all of my attempts ended in
  failure. Essentially the issue is a functional dependency in the MonadState
  typeclass, because of that dependency it becomes mandatory to describe how a
  value is returned from the data type. This is unclear to me; we would need to
  say, given a type TEval, we can uniquely determine (TEval Prims) from it. But
  to instantiate MonadState we need to provide definitions for the \textbf{get}
  and \textbf{put} functions. The \textbf{get} function does not take any
  arguments so we have no way of pattern matching the embedded monad, and
  similarly the \textbf{put} function only takes a state, but we still have no
  way of accessing the underlying monad. Even if pattern matching was possible
  it is not clear how, given the current definition of TEval, one would write
  this. I'm not suggesting this is impossible, but it is much harder to achieve
  than I originally thought. 

  Another design decision that encourages strict evaluation is the use of a
  homogeneous map. I chose to use a homogeneous map because it is closer to
  something I would consider safe in haskell, and because I wanted to stress
  test the tagless final approach. Heterogeneous maps do exist in haskell, but I
  consider them to not be in the spirit of haskell, and furthermore to be fairly
  exotic. Even if a heterogeneous map solves some of the problems stated here it
  would still, in my view, be a mark against the final tagless approach
  precisely because I needed to reach for such an exotic library. In any case,
  the use of a homogeneous map forces me to evaluate everything to a primitive
  before storing it, because the types of the values in the map must be the
  same, in this case the sum type Prims or NewPrims. If this were not the case
  then I would conceivably be able to implement the WHILE language without a the
  Prims datatype and thus any type in haskell could be fair game for use in the
  embedded language.

  \subsection{Adding Lists, one, at, a, time.}
  In the \underline{Extensions.hs} file, in the experiments section I tried to add a
  polymorphic list to the WHILE language using patterns that worked for other
  types:
  
  \begin{verbatim}
    class Lists l where
      llit :: l a -> l [a]
      cons :: l a -> l [a] -> l [a]
      nth  :: l Int -> l [a] -> l a
      head :: l [a] -> l a
      tail :: l [a] -> l [a]
      map_ :: (a -> b) -> l [a] -> l [b]
      rightFold :: (a -> b -> b) -> l b -> l [a] -> l b
  \end{verbatim}

  Similar to other types we start with a literal constructor, in this case it
  just takes something of type ``a'' and wraps it into a list and passes it to
  the ``l'' constructor to be part of the language. Almost all of the other
  functions follow in the same manner and have familiar types wrapped with the
  ``l'' type constructor. The problem is that this definition can never be
  instantiated, because we have no way to dispatch on the incoming ``a'' to
  select the right newPrim value constructor. In a tagged approach this definition
  would also be impossible. But it is important to note that in that approach
  the solution is more terse than the final tagless approach. First we would add
  the appropriate constructor to the language, this is also required in the
  final tagless approach:
  
  \begin{verbatim}
   data NewPrims = NI Int
                 | NB Bool
                 | S String -- extensions in constructor dimension
                 | F Float
                 | L [NewPrims]
                 | Skip
  \end{verbatim}

  Then when we write our semantics we would just pattern match on the list and
  handle it appropriately. In the final tagless approach we cannot pattern
  match. So instead we must make a list class that takes only the types we want.
  Essentially we want this:

  \begin{verbatim}
  class IntLists l where
    ilit  :: NewPrims -> l [NewPrims]
    icons :: NewPrims -> l [NewPrims] -> l [NewPrims]
    ihead :: l [NewPrims] -> l NewPrims
    itail :: l [NewPrims] -> l [NewPrims]
  \end{verbatim}

  This works but my reservation is that it is not really in line with the
  final tagless approach. Basically it is leveraging the tagged primitives used
  in the language to add support for lists of primitives in the language. I'm
  not sure if this is actually ``legal'' in the final tagless style but I chose
  to avoid it. Rather then use, or abuse NewPrims as shown above I show how one
  can add lists to the language, and as an example I use integers. This approach
  more in the spirit of tagless final is shown in the extensions file but I'll
  present it here, as well:
  \begin{verbatim}
  class IntLists l a where
    ilit  :: a -> l [a]
    icons :: l a -> l [a] -> l [a]
    ihead :: l [a] -> l a
    itail :: l [a] -> l [a]
  \end{verbatim}

  Basically we just abstract the type of the element into a type parameter
  ``a''. Then we can specify the type appropriately, dispatch on the right value
  constructor and maintain the spirt of the final tagless approach. The downside
  to this approach is that we will need to write more instances for each type
  that is legal in a list, effectively increasing our boilerplate count even
  more. Also note that this approach necessitates homogeneous lists, while the
  NewPrims driven one shown earlier can create heterogeneous lists.

  Lastly, I want to mention that adding lists to the WHILE language in a final
  tagless approach necessitates that one cannot create hand-rolled loops for
  lists because we cannot pattern match. While this may not seem like a big
  problem it is relatively easy to imagine cases where hand-rolled loops may be
  easier to understand than nested folds and maps. Again, It is not impossible
  to do this in a final tagless way, its just an associated cost with the style
  that I wanted to highlight.
  
  \subsection{Attempts at adding Tagging, Goto/Gosub statements}
  Adding tagging, goto or gosub statements to the language suffers from a few
  complex matters. First, one could model a gosub statement (a statement that
  jumps to some labeled block) with a let statement whose body itself is a
  statement. On the evaluation of a gosub one would only need to pass the
  current state to the code block and then transfer control flow. This can be
  done, but because I have implemented strict evaluation it is basically
  impossible. If we could store statements with the let statement then it may
  be possible. A goto statement would be similar but take advantage of the line
  number instead of a named label, and then pass control flow to that line
  number. The issue with both of these is the vague notion of transferring
  control flow. How does one go about doing this in a tagless final approach? In
  the tagged approach we would recursively call our \textit{eval} function on
  the statement returned by the gosub lookup. For the goto we would need a list
  of linenumbers and statements, using that we could lookup the line number,
  truncate everything that should be skipped, map over the list to remove the
  line numbers and then recursively call eval on the resulting list of
  statements. In the tagless approach we have no way to recursively call our
  eval function because it does not exist. We could return the statement
  returned by the gosub lookup, or we could perform the same process described
  earlier with goto and then return the sequence of statements. Either way we
  don't actually evaluate to a primitive, and we have no way to do so. 

  Tagging statements suffers from similar problems. Typically one is able to tag
  an abstract syntax tree to simulate or determine line numbers in the tree.
  This is done by pattern matching and crawling down the tree transforming it to
  a list of line numbers and their associated statements. For the final tagless
  approach we have no way to pattern match on any statement, so traversing the
  AST becomes problematic because the AST is not just a series of nested
  function calls. Assuming that was solved, and we could traverse the tree, we
  would still have no way to crawl down sub-branches of say an if-statement. For
  an if-statment we would need to crawl down the then branch, retrieve the most
  recent line number, and then crawl down the else branch. However, we have no
  way of discerning an if-statement from any other statement, because we cannot
  pattern match. Again, this may be possible, but in the final tagless approach
  it is definitely not easily done.

  \section{Conclusion}
  In sum, the tagless final approach is quite clean but not without its own set
  of tradeoffs. In general I found that when I stayed close to simple
  extensions, like adding strings, floats and exponentials to the language it
  was quite easy. But even the slightest bit of complexity, like adding a list,
  led to an explosion in design concerns and considerations. My general take
  away is that the tagless final approach is very well suited to certain EDSLs,
  but once one starts running up against some of its rougher edges it becomes
  much more unwiedly. In the beginning of this project I was thinking something
  along the lines of ``I'm a good functional programmer, I can survive without
  pattern matching'', now, on the other side of the project, my view is basically
  that pattern matching isn't always essential, but having it is always
  desirable, and not being able to pattern match is a much greater
  cost than I originally thought.

\end{document}